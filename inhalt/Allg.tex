\section{Allgemein}
\subsection{Symbole}
Stichprobenstandardabweichung $ \hat{=} $ s;
Standardabweichung $ \hat{=} \sigma $
\subsection{Abl.}
$ x^n  \hat{=} nx^{n-1}$
\hrule
$ sin x \hat{=} cos x $; 
$ cos x  \hat{=} - sin x $; 
$ tan x \hat{=} \frac{1}{cos^2 x} = 1 + tan^2 x $;
$ cot x \hat{=} - \frac{1}{sin^2 x} = - 1 -cot^2 x $; 
\hrule
$ e^x \hat{=} e^x $; 
$ a^x \hat{=} (\ln a) \cdot a^x $; 
\hrule
$ \ln x \hat{=} \frac{1}{x} $;
$ \log_a x  \hat{=} \frac{1}{(\ln a) \cdot x}$; 
\hrule
\subsection{Abl.Regeln}
\textbf{Faktorregel} $ y = C \cdot f(x) \Rightarrow  y' = C \cdot f'(x) $; 
\textbf{Summenregel} $ y = f_{1}(x) + f_{2}(x) + ... +f_{n}(x) \Rightarrow y' = f'_{1}(x) + f'_{2}(x) + ... + f'_{n}(x) $; 
\textbf{Produktregel} $ y = u \cdot v \Rightarrow y' = u'\cdot v + v' \cdot u $; 
$ y = u \cdot v \cdot x \Rightarrow y' = u' \cdot v \cdot w + u \cdot v' \cdot w + u \cdot v \cdot x'$; 
\textbf{Quotientenregel} $ y = \frac{u}{v} \Rightarrow y' = \frac{u' \cdot v - u \cdot v'}{v^2}$; 
\textbf{Kettenregel} $ f'(x) = F'(u) u'(x) \hat{=} F'(u): \text{Ableitung der Äußeren Funktion}; u'(x): \text{Ableitung der Inneren Funktion}$
\subsection{Integralregel, elementar}
\textbf{Faktorregel}$\int_{a}^{b}  C \cdot f(x) dx = C \cdot \int_{a}^{b} f(x)dx$; 
\textbf{Summenregel} $ \int_{a}^{b} [f_{1}(x) + ... + f_{n}(x)] dx = \int_{a}^{b} f_{1}(x) dx + ... + \int_{a}^{b}f_{n}(x) dx$; 
\textbf{Vertauschungsregel}$ \int_{b}^{a}f(x) dx = -\int_{a}^{b} f(x) dx$; 
$ \int_{a}^{a} f(x) dx = 0 $; 
$ \int_{a}^{b}f(x) dx = \int_{a}^{c} f(x) dx + \int_{c}^{b} f(x) dx  \text{ für } (a \le c \le b) $; 
\subsection{Berechnung best. Integr.}
$ \int_{a}^{b} f (x) dx = [F(x)]_{a}^{b} = F(b)- F(a) $
\subsection{Potenzen}
$ x^{-n} = \frac{1}{^n} $; 
$ a^0 = 1, a^{-n} = \frac{1}{a^n} $; 
$ a^m \cdot a^n  = a^{m+n} $; 
$ \frac{a^m}{a^n} = a^{m-n} $ für $ a \ne 0 $; 
$ ! (a^m)^n = (a^n)^m = a^{m \cdot n} $; 
$ a^n \cdot b^n = (a \cdot b)^n $; 
$ \frac{a^n}{b^n} = (\frac{a}{b})^n \text{ für } b \ne  0 $; 
$ a >0 : a^b = e^{b \ln a} $; 
$ 0^0 = 1 $; 
$ x_{1}^1 = x_{1} $; 

\subsection{Wurzel}
$ \sqrt{a ^2} = |a| $;
$ b = a^n  \Leftrightarrow a = \sqrt[n]{b}$; 
$ \sqrt[n]{a} = a^{\frac{1}{n}} $; \\
$ \sqrt[n]{a \pm b} \ne \sqrt[n]{a} \pm \sqrt[n]{b} $ \\
\hrule
$ \sqrt[n]{a^m} = (a^m)^{\frac{1}{n}} = a^{\frac{m}{n}} = (a^\frac{1}{n})^m = (\sqrt[n]{a})^m $ \\
$\sqrt[m]{\sqrt[n]{a}} = \sqrt[m]{a^{\frac{1}{n}}} = (a^{\frac{1}{n}})^{\frac{1}{m}} = a^{\frac{1}{m \cdot n}} = \sqrt[m \cdot n ]{a} $\\
$ \sqrt[n]{a} \cdot \sqrt[n]{b} = (a^{\frac{1}{n}}) \cdot (b^{\frac{1}{n}})  = (ab)^{\frac{1}{n}} = \sqrt[n]{ab} $\\
$ \frac{\sqrt[n]{a}}{\sqrt[n]{b}} = \frac{a^{\frac{1}{n}}}{b^{\frac{1}{n}}} = (\frac{a}{b})^{\frac{1}{n}} = \sqrt[n]{\frac{a}{b}} \text{ für } b > 0 $\\
$\Rightarrow m , n \in \mathbb{N}^*; a \ge  0, b \ge 0 $
\subsection{Abc-Formel}
$ x_{1,2} = \frac{-b \pm \sqrt{b^2 - 4ac}}{2a} $; 
$ x_{1,2} = \frac{2a}{-b \mp \sqrt{b^2 -4ac}} $
\subsection{Bin.Formel}
$ (a + b)^2 = a^2 + 2ab + b^2 $ 1. Binom;
$ (a+b)^3 = a^3 + 3a^2b + 3ab^2 + b^3 $; 
$ (a+b)^4 = a^4 + 4a^3b + 6a^2b^2 + 4ab^3 + b^4$
\hrule
$ (a-b)^2 = a^2 - 2ab + b^2 $; 2. Binom;
$ (a-b)^3 = a^3 - 3a^2b + 3ab^2 - b^3$; 
$ (a-b)^4 = a^4 - 4a^3b + 6a^2b^2 - 4ab^3 + b^4 $
\hrule
$ (a+b) (a-b) = a^2 - b^2 $ 3. Binom; 

\subsection{Einigungen}
	$\circ$ Beim Runden mind. eine Nachkommastelle.
\subsection{Trigonometrischer Pythagoras}
$ \sin^2 x + cos^2 x = 1 $
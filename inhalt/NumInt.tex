\section{NumInt}
Verbesserung der Näherung: Aufteilung in kleine Teilintervalle \& Summe von Rechtecksflächen bilden; Interpolations mit Polynom höheren Gredes durch diskrete Punkte.
\subsection{Def}
$ p_{k} \hat{=} $ Interpolationspolynom

\subsection{Newton-Cotes}
Das Intergral des $ p_{k} $ diens al Appr. für das Int. von f(x); 
$ \int_{0}^{1} f(t) dt \approx \int_{0}^{1} p_{k} (t) dt =  \sum_{j=0}^{k} \alpha_j f( t_{j} ) $ Das Interpolationspolynom muss nicht explizit aufgestellt werden, es dient vorab der Bestimmung der Gewichte $ \alpha_{j} $; 
$ \int_{0}^{1} p_{k}(t) = \int_{0}^{1} \sum f( t_{j} ) L_{j} (t) dt = \sum f ( t_{j} ) \int_{0}^{1} L_{j} (t) dt $
\subsubsection{Trapezregel}
$ T_{1}: $ 
$ \int_{0}^{1} f (t) dt \approx \frac{1}{2} ( f(0) + f(1) ) $; 
$ \int_{a}^{b} f (x) dx \approx \frac{(b-a)}{2} (f(a) + f(b) ) $;\\
$ T_{n}: $
Für Teilintervalle mit gleicher Länge: $ h = \frac{b-a}{n} $;  
$ T_{n} = h (\frac{ f ( x_{0} ) }{2} + f(x_{1}) + ... + f(x_{n-1}) + \frac{ f(x_n) }{2} ) $; 
\subsubsection{SimpsonRegel}
$ S_{1}: $ 
$ \int_{0}^{1} f (t) dt \approx \frac{1}{6} ( f(0) + 4f(0.5) + f(1) ) $; 
$\int_{a}^{b} f (x) dx \approx \frac{b-a}{6} ( f(a) + 4f( \frac{a+b}{2}) + f(b) ) $; 
Für n = 1: $\frac{( b-a )}{2\cdot1} \frac{1}{3} ( f(a) + 4f( \frac{a+b}{2}) + f(b) ) $; 
Für n allg.: $ \frac{( b-a) }{2n}\frac{1}{3} ( f(a) + 4(a+h) + ... + 4f(b-h)+ f(b) ) $ 
$ S_{n}:$ 
\textbf{Beachte gerade Anzahl an Teilinvervallen!}; 
Für 2n Teilintervalle, 2n+1 Knoten mit gleicher Länge $ h = \frac{ b-a }{2n} $; 
$ S_{2} = \frac{h}{3} ( f(x_{0}) + 4f(x_{1}) + 2f(x_{2}) + 4f(x_{3}) + f(x_{4}) ) $; 
\includegraphics[scale=0.25]{./pic/NewtonCodesRegeln.png}
Falls $\alpha_{i} $ positiv. Integrationsregeln stabil; 
$ k \le 7 \& k =9 \Rightarrow $ positive Gewichte;
Bei halbierung der Intervalle \textbf{Nachfrage} vervierfacht or versechszehnfacht sich der Fehler?
\subsection{FehlerQuadratur}

\section{Fehleranalyse}
\subsection{Auslöschung}
wenn ungefähr gleich große, bereits mit Fehlern behaftete Zahlen voneinander abgezogen werden \& signifikante Mantissenstellen ausgelöscht werden.
\subsection{Addition}
große signifikante Stellen schlucken kleine signifikante Stellen.
\subsection{Horner}
Ohne: Runden bei jeder Rechenoperation. Mit: Vermeidung der Rundungsfehler nach jeder Rechenoperation.
\subsection{Abc-Formel}
$ x_{1,2} = \frac{-b \pm \sqrt{b^2 - 4ac}}{2a} $; 
$ x_{1,2} = \frac{2a}{-b \mp \sqrt{b^2 -4ac}} $; 
b>0, dann (2), für $ x_1 $ \&(1) für $ x_2 $ or 
b<0, (1) für $x_{1} $ \& (2) für $ x_2 $;
Falls 4ac klein im Vergleich zu $ b^2$, dann evtl. Probleme der Auslöschung.
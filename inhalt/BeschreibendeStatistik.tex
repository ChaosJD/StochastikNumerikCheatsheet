\section{BeschreibendeStatistik}
  \subsection{Begriffe}]
    \subsubsection{Beschreibende/Deskriptive Statistik}
    Beobachtete Daten werden durch geeignete statistische Kennzahlen charakterisiert und durch geeignete Grafiken anschaulich gemacht.
    \subsubsection{Schließende/Induktive Statistik}
    Aus beobachtete Daten werden Schlüsse gezogen und diese im Rahmen vorgegebener Modelle der Wahrscheinlichkeitstheorie bewertet.
    \subsubsection{Grundgesamtheit}
    $\Omega$: Grundgesamtheit
    $\omega$:Element oder Objekt der Grundgesamtheit
    diskret(<30 Ausprägungen), stetig($\geq$30 Ausprägungen), univariat(p=1), mulivariat(p>1)
  \subsection{Lagemaße}
    \subsubsection{Modalwerte}
    Am häufigsten auftretende Ausprägungen (insbesondere bei qualitativen Merkmalen)
    \subsubsection{Mittelwert}
    $\overline{X} = \frac{1}{n} \sum_{i=1}^{n} x_{i}$
    \subsection{Median}
    \begin{equation}
      x_{0.5} =
        \begin{cases}
        	x_{\frac{n+1}{2} \text{, falls n ungerade}}\\
        	\frac{1}{2}(x_{\frac{n}{2}} + x_{\frac{n}{2}+1}) \text{, falls n gerade}
        \end{cases}
    \end{equation}
  \subsection{Streuungsmaße}
\section{Hypothesentests}
Basierend auf n unabhängig und identisch Verteilte (i.i.d) Zufallsvariablen $ X_{1}, ..., X_{n} $(Messungen) soll eine Entscheidung getroffen werden, ob eine Hypothese für einen unbekannten Erwartungswert $ \mu $ gültig ist or nicht.
\subsection{Def}
$\alpha $ = Signifikanzniveau/ Fehlerwahrscheinlichkeit 
TG = Prüfgröße; 
TG* = standardisierte Prüfgröße; 
siginifikante Schlussfolgerung = $ H_{0} $ verworfen $\rightarrow$ klassischer Parametertest; 
schwache Schlussfolgerung = $ H_{0} $ wird nicht verworfen $\rightarrow$ klassischer Parametertest.
p-Wert = beobachtetes Signifikanzniveau
\subsection{Null- und Gegenhypothese}
\textbf{Modell:} Verteilung der Grundgesamtheit or Testgröße \textbf{TG} ( häufig $\overline{x}$ ) ist bekannt bis auf einen Parameter, z.B. $ \mu $, für den eine Hypothese aufgestellt wird.
$ TG \sim  N_{\mu, \sigma^2}$; 
\textbf{Nullhypothese: $ H_{0}$:} Angezweifelte Aussage, der widersprochen werden kann, wenn die Stichprobe einen Gegenbeweis liefert. $ H_{0}: \mu = \mu_{0}$; 
\textbf{Gegenhypothese $ H_{1} $:} Gegenteil von $ H_{0} $ z.B. $ H_{1} \neq \mu_{0} $;
\subsection{Ablehnungsbereich, Fehler 1. \& 2.}
Treffen der Testentscheidung, basierend auf einer konkreten Stichprobe 
$ \{x_{1}, ..., x_{n} \} $; Berechnung der Realisation $ tg = TG(x_{1},..., _x{n}) $ der Prüfgröße TG; 
\textbf{Ablehnungsbereich / Kritischer Bereich C}: Werte der Testgröße, die für H1, sprechen \& bei Gültigkeit von $ H_{0} $ mit Wahrscheinlichkeit $ \le \alpha $ ( meist 0.1, 0.05, or 0.01) auftreten.\textbf{Fehler 1. Art:}$ \alpha $ ist die Wahrscheinlichkeit, dass $ H_{0} $ verworfen wird, obwohl sie richtig ist.
\textbf{Annahmebereich:} Komplement $ \overline{C} $ des Ablehnungsbereichs. $ H_{0} $ kann nicht abgeleht werden, falls $ tg \in \overline{C} (P(tg \in \overline{C}) \ge 1 - \alpha) $.\textbf{Fehler 2. Art:} Die Wahrscheinlichkeit, dass $ H_{0} $ nicht abgelehnt wird, obwohl sie falsch ist.
\includegraphics[scale=0.25]{./pic/Testzenarien.png}
\includegraphics[scale=0.25]{./pic/StandardnormvalverteilungTestgroese.png}
$H_{0}: \mu = \mu_{0} $; $ H_{1}: \mu \neq \mu_{0} $; 
\subsection{Klassischer Parametertest}
$H_{0} $ wird abgelehnt, falls $ tg = TG( x_{1}, ..., x_{n} ) \in C $; 
$ H_{0} $ wird angenommen falls $ tg = TG(x_{1}, ..., x_{n}) \in \overline{C} $; 
Der kritische Bereich ergibt sich analog zu den Konfidenzintervallen durch die Vorgabe eines kleinen Signifikanzniveau $ \alpha $ d.h. max. Wahrscheinlichkeit für Fehler 1. Art, mit standardisierter Prüfgröße TG* gilt:
$ P (TG \in C) \le \alpha  \Leftrightarrow TG^{*} \in ]-\infty; \phi^{-1}(1- \frac{\alpha}{2}) [ \cup ] \phi^{-1}(1-\frac{\alpha}{2}); \infty [$; 
$ P(TG \in \overline{C}) \ge 1 - \alpha \Leftrightarrow TG^{*} \in [ \phi^{-1}(\frac{\alpha}{2}), \phi^{-1}(1-\frac{\alpha}{2}) ] $; 
Wird dann $ H_{0} $ verworfen, spricht man von einer signifikanten Schlussfolgerung. Kann $ H_{0} $ nicht verworfen werden, dann lässt sich keine Aussage über den Fehler 2. Art treffen \& man spricht von einer schwachen Schlussfolgerung.
\subsection{Zweiseitiger Gauß Test}
$ H_{0}: \mu = \mu_{0}  $ gegen $ H_{1}: \mu \neq \mu_{0} $; 
$ \overline{ X } \sim  N_{ \mu_{0}, \sigma_{0}^2 /n} \Rightarrow \frac{ \overline{X} - \mu_{0} }{ \sigma_{0} } \sqrt{n} \sim N_{0, 1} $; 
$ P_{ \mu0}( \overline{X} \in C ) \le \alpha \Leftrightarrow |TG| = \frac{ | \overline{X} - \mu_{0} | }{ \sigma_{0} } \sqrt{n} > \phi^{-1}(1-\frac{ \alpha}{2} )$; 
\textbf{Testentscheidung:} $ H_{0} $ wird abgelehnt, falls $ |TG| > \phi^{-1}(1-\frac{ \alpha }{ 2 } ) $; $ H_{0} $ wird angenommen, falls $ |TG| \le \phi^{-1}(1-\frac{ \alpha }{2} )$
\subsection{Einseitiger Gauß Test}
\subsubsection{linksseitig}
$ H_{0}: \mu \ge \mu_{0} $ gegen $ H_{1}: \mu  < \mu_{0}$
\subsubsection{rechtsseitig}
$ H_{0}: \mu \le \mu_{0} $ gegen $ H_{1}: \mu > \mu_{0} $
\hrule
$ P_{\mu 0} ( \overline{X} \in C ) \le \alpha \Leftrightarrow TG = \frac{ \overline{ X } - \mu_{0} }{ \sigma_{0} } \sqrt{n} < \phi^{-1} ( \alpha ) $; 
\textbf{Testentscheidung:} $ H_{0} $ wird abgelehnt falls, $ TG < \phi^{-1} ( \alpha )$; 
$ H_{0} $ wird angenommen, falls $ TG \ge \phi^{-1} ( \alpha ) $; 
\includegraphics[scale=0.25]{./pic/EinseitigerGausTest.png}
\subsection{Varianten Gauß Test, $ \sigma^2 $ bekannt, $\mu$ unbekannt}
Prüfgröße$ tg = \frac{ \overline{X} - \mu_{0} }{ \sigma_{0} }\sqrt{n} $; \\
\includegraphics[scale=0.25]{./pic/GausTests.png}
\subsection{t-Test, $ \mu, \sigma^2 unbekannt  $}
Prüfgröße $ tg = \frac{ \overline{X} -\mu_{0} }{ S } \sqrt{n} \sim t_{n-1} $
\includegraphics[scale=0.25]{./pic/tTest.png}
\subsection{p-Wert}
Wahrscheinlichkeit, bei Zutreffen von $ H_{0} $ den beobachteten Wert tg der Prüfgröße or einen noch stärker von $ \mu_{0} $ abweichenden Wert zu bekommen.
%Grafik p-Value slide 15+1?
Der p-Wert zu einer Hypothese $ H_{0} $ ist der kleinste Wert von $ \alpha $, für den $ H_{0} $ noch abgelehnt werden kann. \textbf{Je kleiner} der Wert, \textbf{desto kleiner} ist der Fehler 1. Art \& umso signifikanter ist die Testentscheidung. 
\textbf{Nice to know} Anhand des p-Werts kann man für beliebige Werte von $ \alpha $ eine Testentscheidung treffen;\\
Falls $ p-Wert < 1\%:$ sehr hohe Signifikanz\\
Falls $ 1\% \le p-Wert < 5\%: $ hohe Signifikanz\\
Falls $ 5\% \le p-Wert \le 10\%: $ Signifikanz\\
Falls $ p-Wert > 10\%: $ keine Signifikanz\\
\subsection{Zusammenhang I \& Hypothesentests zweiseitig}
zum Konfidenzniveau $ 1- \alpha $; 
$ H_{0} $ wird abgelehnt, falls $ \mu_{0} \notin I $; 
$ H_{0} $ wird angenommen, falls $ \mu_{0} \in I $; 
Das Konfidenzniveau ist der Annahmebereich von $ H_{0} $ zum Signifikanzniveau $ \alpha $; 
\subsection{Zusammenfassung klass. Hypo.test}
Signifikanzniveau $ \alpha $ wird vorgegeben;\\
$ \alpha $ \& Verteilung der Testgröße unter $ H_{0} $ wir der Ablehnungsbereich ermittelt. \textbf{Je kleiner (größer) } $ \alpha $ , \textbf{desto kleiner (größter) } ist der Ablehnungsbereich; \\
\textbf{!:} $ \alpha \& C $  hängen \textbf{nicht von} der konkreten Stichprobe ab;\\
$ H_{0} $ wird abgelehnt, falls der ermittelte Wert der Testgröße (beobachteter Wert) in C liegt. 
\textbf{!:} Die tg hängt von der konkreten Stichprobe ab. Sie ist eine ZV.
\subsection{Test mittels p-Wert}
$ \alpha $ wird vorgegeben.\\
Berechnung des p-Werts anhand der konkreten Stichprobe mit der Verteilung der Tg unter $ H_{0}  $;\\
\textbf{!:}Der p-Wert hängt von der konkreten Stichprobe ab, ist eine ZV.\\
$ H_{0} $ wird abgelehnt, falls $ p-Wert \le \alpha. $;


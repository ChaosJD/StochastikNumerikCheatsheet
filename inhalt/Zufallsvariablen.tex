\section{Zufallsvariable}
Abbildung des \textbf{abstrakte} Ergebnisraums $\Omega$ auf $\mathbb{R}$.
Eine Abbildung $X: \Omega \rightarrow \mathbb{R}$, $\omega \mapsto X(\omega)$ = heißt Zufallsvariable (ZV). x $\in$ R. heißt Realisation der ZV X.
  \begin{itemize}
    \item Diskrete ZV: $X(\Omega) = {x_{1}, ..., x_{2}} (n \in \mathbb{N})$; z.B. X = \dq Augensumme beim Würfeln \dq
    \item Stetige ZV: $X(\Omega) \subseteq \mathbb{R}$; \dq z.B. Körpergröße eines Menschen\dq
  \end{itemize}
\subsection{Verteilungsfunktion-allg.}
Die Wahrscheinlichkeit P(B) für ein Ereignis B in $\mathbb{R}$ wird zurückgefürht auf die Wahrscheinlichkeit der entsprechenden Ereignisse in $\Omega$. Für jedes $X \in \mathbb{R}$  ist die Verteilungsfunktion F: $\mathbb{R} \rightarrow [0,1]$ einer ZV X definiert durch:\\
	F(x) = P(X $\leq$ x)
\begin{itemize}
	\item $0 \leq F(x) \leq 1$
	\item $\displaystyle\lim_{x\to-\infty} F(X)= 0 \, \lim_{x\to\infty} F(x) = 1$
	\item monoton wachsend
	\item $P(X > x) = 1 - F(x)$
	\item $P(a < X \leq b) = F(b) - F(a)$
\end{itemize}
\subsection{Diskrete ZV}
Für eine diskrete ZV X mit $X(\Omega) = {x_{1}, ..., x_{n}}$ ( n endlich oder abzählbar unendlich) ist die Wahrscheinlichkeitsfunktion definiert durch:\\
\begin{equation}
p(x) =
\begin{cases}
	P(X = x_{i}), \text{falls } x_{i} \in X(\Omega )\\
	0, sonst\\
\end{cases}
\end{equation}
\textbf{Es gilt:}
\begin{itemize}
	\item $F(x) = (P(X \leq x) = \sum_{x_{i}\leq x} p(x_{i})$
	\item F(x) ist eine rechtseitig stetige \textbf{Treppenfunktion} mit \textbf{Sprüngen} bei der Realisation von $x_{i}$.
\end{itemize}
\subsection{Stegite ZVs}

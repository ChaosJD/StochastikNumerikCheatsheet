\section{Konfidenzintervall}
kl. Stichpr.umf. (n<30) ist die Grundgesamtheit näherungsweise normalverteilt or Stichpr.umf. ist hinreichend groß (n≥30), die Sum. or. der Mittelwert der $ X_i $ nach dem ZGWS näherungsweise norm.vert. ist
  \subsection{Begriffe}
   Irrtumswahrscheinlichkeit = $ \alpha $; 
  Konfidenzniveau =  $ 1-\alpha $ ; 
  Konfidenzintervall = $ I $
  \subsection{Punkschätzer}
  E[X]: Stichprobenmittel:
  $ \overline{X = \frac{1}{n}} \sum_{i=1}^{n} X_{i}$; 
  Varianz: Stichprobenvarianz:
  $ s^{2} = \frac{1}{n -1} \sum_{i=1}^{n} (X_{i} - \overline{X})^{2} $; 
  Schätzwert für wahren Parameter, aber keine Aussage über Unsicherheit der Schätzung, Geringe Sicherheit für wahren Parameter; 
  
  \subsection{Intervallschätzer}
  Intervall für wahren Parameter, mit vorgegebener Sicherheit; Vorgabe (95\% or 99\%); 
  Dichtefunktion:
  \includegraphics[scale=0.25]{./pic/KonfidenzintervallDichtefunktion.png}
  $ P(-a \le \overline{x} \le a) > 0.95 $; 
  $ \sigma ist $ unbekannter Parameter\\
  $ P( x_{\underbrace{0.025}} < \underbrace{ \frac{\overline{x} - \mu}{\sigma}\sqrt{n} } < x_{\underbrace{0.975}} ) \ge 0.95 $\\
  $ -1.96; N_{0,1}; 1.96 $; 
  
  \subsection{$ \mu $, unbekannt, $ \sigma^{2} $, bekannt}
  $ I = ] \overline{X} \textbf{-} \underbrace{\phi^{-1}( 1-\frac{\alpha}{2} ) } \frac{\sigma}{ \sqrt{n} }\textbf{,} $\\ 
  $ qnorm ( 1-\frac{\alpha}{2} ) $\\
  $ \overline{X} \textbf{+} \phi^{-1} ( 1- \frac{\alpha}{2} ) \frac{\sigma}{ \sqrt{n} } [ $; 
  \includegraphics[scale=0.25]{./pic/QnormTabelle.png}
  \includegraphics[scale=0.25]{./pic/KonfidenzintervallDichtefunktionTabelle.png}

  \subsection{$ \mu $ \& $ \sigma^{2} $, unbekannt }
  $ I = ] \overline{X} \textbf{-} t_{n-1}^{-1} ( 1-\frac{\alpha}{2} ) \frac{S}{ \sqrt{n} } \textbf{,} \overline{X}  \textbf{+} t_{n-1}^{-1} ( 1-\frac{\alpha}{2} )\frac{S}{ \sqrt{n} } [ $
  \subsection{Zusammenfassung}
  Wie verändert sich das $ (1 - \alpha ) $-Konfidenzintervall, n-größer $\Rightarrow$ I kürzer; 1-$ \alpha $ größer $ \Rightarrow $ I länger; 
  Für $ \frac{L}{2} = 2 \phi^{-1}( 1-\frac{ \alpha}{2} ) \frac{\sigma}{ \sqrt{n} } \frac{1}{2} = 2 \phi^{-1} ( 1- \frac{ \alpha }{2} ) \frac{\sigma}{ \sqrt{4n} } $
  \subsection{Aufgabentypen}
  \textbf{Geg: } n, 1-$ \alpha $; 
  \textbf{Ges: }I s.o.
  \textbf{Geg: }$ \overline{X}, \sigma, 1- \alpha, L $; 
  $ L = 2 \phi^{-1} ( 1-\frac{ \alpha }{2} ) \frac{\sigma}{ \sqrt{n} }$; 
  \textbf{Ges: }n; $ \sqrt{n} > 2\phi^{-1} ( 1- \frac{ \alpha}{2} )\frac{ \sigma }{L} $
  \textbf{Geg: }n, I, L; 
  \textbf{Ges: }1- $\alpha$; 
$ 1- \frac{ \alpha }{2} = \phi (\frac{ L \sqrt{n} }{ 2\sigma} ) $
  